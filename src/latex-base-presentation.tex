\documentclass{beamer}

\input{header-presentation}
\usepackage{graphicx}
\graphicspath{{../img/}}
\title[Group Meeting Presentation]{Quantifying and Improving the Efficiency of Hardware-based Mobile Malware Detectors}
\author[bobzhou@bu.edu]{Mikhail Kazdagli, Vijay Janapa Reddi, Mohit Tiwari\\
Presented by Boyou Zhou}
\date[\today]{\today}

\begin{document}
\maketitle

\section{High Level Summary}

\begin{frame}{Arch + Software Defense}
	\begin{columns}[b]
		\column{0.5\textwidth}
		\begin{block}{Hardware for Security}
			\begin{itemize}
				\item To overcome semantics gaps to identify malwares on the hardware level
				\item To ''understand'' the binary with machine learning algorithms
				\item To classify malwares from benignwares
			\end{itemize}
		\end{block}
		
		\column{0.5\textwidth}
		\begin{figure}
			\centering
			\includegraphics[width=\textwidth]{overview_of_Sherlock.png}
			\caption{Overview of Sherlock}
		\end{figure}
	\end{columns}
\end{frame}

\begin{frame}{Hardware Performance Counters}
	\begin{block}{Academic Approaches}
		\begin{itemize}
			\item Supervised learning\footnote[frame]{Demme ISCA'13, Khasawneh RAID'15}
			\item Unsupervised learning\footnote[frame]{Tang RAID'14}
			\item Hardware implemented\footnote[frame]{Ozsoy HPCA'15}
		\end{itemize}
    \end{block}
	\begin{block}{Industrial Approach}
		\begin{itemize}
			\item Qualcomm snapdragon\footnote[frame]{https://www.qualcomm.com/products/snapdragon/security/smart-protect}
		\end{itemize}
	\end{block}
\end{frame}

\begin{frame}{Design Targets}
	\begin{columns}
		\column{0.6\textwidth}
		\begin{block}{Traditional designs}
			\begin{itemize}
				\item without various delays in malware triggering
				\item without user interactions
			\end{itemize}
		\end{block}
\pause
		\begin{block}{Sherlock designed to detect}
			\begin{itemize}
				\item Malwares that evade detections (static analysis)
				\item Malwares that hide behind benign programs
			\end{itemize}
		\end{block}
\pause

		\column{0.4\textwidth}
		\begin{block}{Sherlock are evaluated}
			\begin{itemize}
				\item with synthesized combination of malware
				\item when benign and malware programs with real user inputs
			\end{itemize}
		\end{block}
	\end{columns}
\end{frame}

\begin{frame}{Metric Explanation}
	\begin{block}{Metric}
		\begin{itemize}
			\item A working operating range that Sherlock can give alert to the
				system.
\pause
			\item An operating range that the smallest malware hidden in
				benignware that Sherlock can detect with certain pre-set false
				positive rate.
\pause
			\item e.g. Firefox got compromised with malware that can still SMSs 
				and contact information when Sherlock system can detect the malwares 
				with pre-set $5\%$ false positive rate.
		\end{itemize}
	\end{block}
\end{frame}

\begin{frame}{Evaluations}
	\begin{itemize}
		\item Unsupervised Learning $\sim24.7\%$ improvements in ROC curve
			(AUC) metric than prior work\footnote[frame]{Tang RAID'14}
		\item Supervised Learning $\sim12.5\%$ improvements in ROC curve
			(AUC) metric than prior work\footnote[frame]{Demme ISCA'13}
		\item Synthesized malware can be detected by HMD but not static
			analysis tool
	\end{itemize}
\end{frame}

\begin{frame}{ROC AUC example}
	\begin{figure}
		\centering
		\includegraphics[width=0.6\textwidth]{true_false_positive_rate.png}
		\caption{ROC(Receiver Operating Characteristic) AUC(Area Under Curve)}
	\end{figure}
\end{frame}

\section{Motivations}
\begin{frame}{Challenges in HMDs(Hardware Malware Detectors)}
	\begin{itemize}
		\item Small changes in deployments of malware will cheat the HMDs. 
			e.g. number of malwares in one application, delays, or change
			of victims
		\item Changes in user inputs will destroy the HMD profile.
	\end{itemize}
\end{frame}


\begin{frame}{Compared to Previous Work}
	\begin{figure}
		\centering
		\includegraphics[width=0.6\textwidth]{malware_comparison.png}
		\caption{Previous work will miss-classified malware. Most of
			previous work\footnote{Demme ISCA'13} experiments last less than 10
			seconds, while stealing a photo may require 3 seconds.  This paper,
			experiments are measured on a much longer time scale.}
	\end{figure}
\end{frame}

\begin{frame}{Compared to Previous Work}
	\begin{figure}
		\centering
		\includegraphics[width=0.6\textwidth]{diff_firefox.png}
		\caption{Benign applications with malware are closer to benignware
			without malware than other benignware.}
	\end{figure}
\end{frame}

\begin{frame}{Compared to Previous Work}
	\begin{figure}
		\centering
		\includegraphics[width=0.6\textwidth]{real_user_input.png}
		\caption{Real user inputs changes will not be able to fit in previous
			work classifiers.}
	\end{figure}
\end{frame}

\section{Malware Synthesizer}

\begin{frame}{Categories of Malware}
	\begin{itemize}
		\item Information Stealer $\sim$ e.g. stealing SMSs, device IDs
		\item Network Nodes $\sim$ e.g. blocking network to fraud
		\item Compute Nodes $\sim$ e.g. running bitcoin miner on compromised machines
	\end{itemize}
\end{frame}

\begin{frame}{Different Classes in Malware}
	\begin{figure}
		\centering
		\includegraphics[width=0.6\textwidth]{malware_classes.png}
		\caption{Classes in malware}
	\end{figure}
\end{frame}

\begin{frame}{Malware behaviors}
	\begin{figure}
		\centering
		\includegraphics[width=0.6\textwidth]{malware_classes2.png}
		\caption{Examples of malware behaviors and their contributions to total
		malware dataset}
	\end{figure}
\end{frame}

\begin{frame}{Different Classes Comparisons}
	\begin{figure}
		\centering
		\includegraphics[width=0.6\textwidth]{malware_class_comparison.png}
		\caption{4 Information Stealer, 2 Network Nodes, and 1 Compute Node}
	\end{figure}
\end{frame}

\begin{frame}{Different App Comparisons}
	\begin{figure}
		\centering
		\includegraphics[width=0.6\textwidth]{app_comparison.png}
		\caption{Real User inputs. Each application traces up to $\sim$ 2 hours.}
	\end{figure}
\end{frame}

\begin{frame}{Malware Synthesizer}
	\begin{itemize}
		\item The payload of the malware will be repackaged into benignware.
		\item Payloads are deployed with different amount of malwares and
			different delays of triggering the malware.
		\item \textit{apktool} is used for decompiling and repackaging. After 
			Repacking, a new application is created.
	\end{itemize}
\end{frame}

\begin{frame}{Performance Counters Choices}
	\begin{itemize}
		\item Memory loads/reads
		\item Immediate control flow instruction counts
		\item Indirect control flow instruction counts
		\item Total number of executed instructions
		\item Integer computations 
	\end{itemize}
\end{frame}

\section{Evaluation}

\begin{frame}{ML algorithms and feature selections}
	\begin{block}{ML algorithms}
		\begin{itemize}
			\item Bag-of-words algorithm $\sim$ Discrete Wavelet Transform to
			extract features (In the paper, there are also comparisons between Discrete
			Wavelet Transform and Power Transform)
			\item Markov Model $\sim$ Bayesian Information Criterion to find
				best 10 $\sim$ 20 states
		\end{itemize}
	\end{block}
\end{frame}

\begin{frame}{Markov Example}

	\begin{figure}
		\centering
		\includegraphics[width=0.33\textwidth]{benign_malware_comparison_3.png}
		\includegraphics[width=0.33\textwidth]{malware_benign_comparison_2.png}
		\includegraphics[width=0.33\textwidth]{malware_benign_comparison.png}
		\caption{Probability means probability of benignity. With higher intensity 
			of malware, the detector can detect malware at an earlier state.}
	\end{figure}

\end{frame}

\begin{frame}{Power Transform Visualization}
	\begin{figure}
		\centering
		\includegraphics[width=0.6\textwidth]{distribution_load_store_evens.png}
		\caption{Clean profile is normalized to Gaussian Distribution.}
	\end{figure}
\end{frame}


\begin{frame}{Different Transform Methods}
	\begin{figure}
		\centering
		\includegraphics[width=0.6\textwidth]{transform_comparison.png}
		\caption{DWT is $\sim24.7\%$ better than power transform.}
	\end{figure}
\end{frame}

\begin{frame}{Bag-of-Words}
	\begin{figure}
		\centering
		\includegraphics[width=0.6\textwidth]{bag_of_words.png}
	\end{figure}
\end{frame}


\begin{frame}{Markov Model}
	\begin{figure}
		\centering
		\includegraphics[width=0.6\textwidth]{markov_model.png}
	\end{figure}
\end{frame}

\begin{frame}{Two Class classifier}
	\begin{figure}
		\centering
		\includegraphics[width=0.6\textwidth]{area_under_the_curve.png}
		\caption{Balanced means that all the malware are labeled as one label.}
	\end{figure}
\end{frame}

\begin{frame}{Random Forest}
	\begin{figure}
		\centering
		\includegraphics[width=0.6\textwidth]{random_forest.png}
	\end{figure}
\end{frame}

\begin{frame}{Thank you}
	Thank you~	
\end{frame}

\end{document}

