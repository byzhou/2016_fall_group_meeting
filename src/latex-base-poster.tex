\documentclass{beamer}
%colorthemes
\usecolortheme{whale}
\usecolortheme{orchid}

\usepackage{booktabs}
\usepackage{multirow}
\usepackage{listings}
\lstset{basicstyle=\footnotesize}
\usepackage[outdir=build/]{epstopdf}
\usepackage{xfrac}
\usepackage{xspace}
\usepackage{amsmath}
\usepackage{hyperref}

%% Tikz related
\usepackage{pgfplots}
\usepgfplotslibrary{external}
\usetikzlibrary{external,pgfplots.groupplots}
\tikzexternalize[prefix=./build/]
\ifpdf
\tikzset{external/system call={latex \tikzexternalcheckshellescape -halt-on-error -interaction=batchmode -jobname "\image" "\texsource" && dvips -o "\image".ps "\image".dvi}}
\else
\fi

%% Colorbrewer
%\input{colorbrewer}

\usepackage{mathtools}
\DeclarePairedDelimiter{\ceil}{\lceil}{\rceil}

\newcommand{\todobase}[3]{{\color{#1}{\textbf{#2[}}{#3}{\textbf{]}}}\xspace}
\newcommand{\todo}[1]{\todobase{Blues9p5}{TODO}{#1}}
\newcommand{\schuyler}[1]{\todobase{Greens9p5}{SCHUYLER}{#1}}
\newcommand{\fixme}[1]{\todo{#1}}
\newcommand{\missingcitation}{[{\color{cyan}\textbf{??}}]\xspace}
\newcommand{\missingdata}{{\color{Reds9p5}\textbf{XX}}\xspace}
% Creates a dummy figure. Modified from:
%   https://gist.github.com/dpgettings/9635856
\newcommand{\missingfigure}[1]{
  \centering
  \fbox{
    \begin{minipage}[c][0.8\columnwidth][c]{0.8\columnwidth}
      \centering{\todo{#1}}
    \end{minipage}
  }
}

\newcommand{\missingwidefigure}[1]{
  \centering
  \fbox{
    \begin{minipage}[c][0.8\columnwidth][c]{0.8\textwidth}
      \centering{\todo{#1}}
    \end{minipage}
  }
}

\lstdefinestyle{customc}{
  belowcaptionskip=1\baselineskip,
  breaklines=true,
  %% frame=L,
  xleftmargin=\parindent,
  language=C,
  showstringspaces=false,
  basicstyle=\footnotesize\ttfamily,
  %% keywordstyle=\bfseries\color{green!40!black},
  %% commentstyle=\itshape\color{purple!40!black},
  %% identifierstyle=\color{blue},
  %% stringstyle=\color{orange},
  %% [TODO] Possible convert everything to colorbrewer colors?
  keywordstyle=\bfseries\color{Greens9p8},
  commentstyle=\itshape\color{Purples9p7},
  identifierstyle=\color{Blues9p8},
  stringstyle=\color{Oranges9p7},
  %% keywordstyle=\bfseries\color{Greens9p8},
  %% commentstyle=\itshape\color{Purples9p7},
  %% identifierstyle=\color{Blues9p6},
  %% stringstyle=\color{Oranges9p5},
}

\lstdefinestyle{customasm}{
   belowcaptionskip=1\baselineskip,
   %% frame=L,
   xleftmargin=\parindent,
   language=[x86masm]Assembler,
   basicstyle=\footnotesize\ttfamily,
   keywordstyle=\bfseries\color{Greens9p8},
   commentstyle=\itshape\color{Purples9p7},
   identifierstyle=\color{Blues9p8},
   stringstyle=\color{Oranges9p7},
   %% commentstyle=\itshape\color{purple!40!black},
   %% keywordstyle=\color{black},
   %% identifierstyle=\color{black},
   tabsize=4,
   %% numbers=left,
   %% numbersep=8pt,
   %% stepnumber=1,
   %% numberstyle=\tiny\color{gray}
}
\lstset{emph={
    accwrite, accread, retq, rax, rbx, rcx, rdx, rsi, rdi
  },emphstyle={\bfseries\color{Greens9p7}}
}

%% \lstset{escapechar=@,style=customc}

\newcommand{\ignore}[1]{}

%paper size
\usepackage[size=custom, width=91.44, height=121.92, scale=1]{beamerposter}
\usepackage{tcolorbox}
\usepackage{svg}
\usepackage{epstopdf}
\usepackage[export]{adjustbox}
\usepackage[font=large,labelfont=large]{caption}
\usepackage{enumitem}
%\usepackage{subfigure}

%posterbox def
\newcommand{\titlebox}{
    \begin{center}
        \begin{tcolorbox}[left=2cm,right=2cm,arc=2cm,boxrule=5mm,width=\textwidth,colframe=blue!70!black,colback=yellow!20]
            \begin{center}
}

\newcommand{\titleboxend}{
            \end{center}
        \end{tcolorbox}
    \end{center}
}

\newcommand{\posterbox}[1]{
    \begin{tcolorbox}[arc=5mm,width=\textwidth,colback=yellow!20,colframe=blue!70!black,adjusted
    title=#1,fonttitle=\LARGE,fontupper=\large,
    ]
    %graphical environment=tikzpicture,
    %frame engine=path,
    %interior titled engine=special,
    %interior engine=special,
    %segmentation engine=special,
    %title engine=path,
    %skin=beamer
}%

\newcommand{\posterboxend}{
    \end{tcolorbox}
}

\begin{document}
    {\color{white}empty empty}\\
    
    \begin{columns}
        \column{0.2\textwidth}
            \begin{figure}
                \includegraphics[width=3in]{img/Boston_University_seal.eps}
            \end{figure}
        \column{0.6\textwidth}
            \titlebox
            \Huge{Detecting Hardware Trojans Using Backside Optical Imaging of
            Embedded Watermarks}\\
            \large{Boyou Zhou, Ronen Adato, Mahmoud Zangeneh, Tianyu Yang, Aydan
            Uyar, \\Bennett Goldberg, Selim Unlu, Ajay Joshi\\}
            \titleboxend
        \column{0.2\textwidth}
            \begin{figure}
                \includegraphics[width=5in]{img/Boston_University_Wordmark.eps}
            \end{figure}
    \end{columns}

    {\color{white}empty empty}\\
    \begin{columns}
        \begin{column}[T]{0.33\textwidth}
            \posterbox{Introduction}
                \begin{itemize}
                    \item [*] CMOS integrated circuits (ICs) become
                    critical in both defense and municipal support systems.
                    \item [*] The IC industry has a trend towards international
                    coorpation and away from vertical integration,
                    which requires the security of ICs.
                    \item [*] During the various phases of the IC integration,
                    the ICs face threats from Hardware Trojans (HTs), IP privacy
                    and IC counterfeiting. 
                    \item [*] Among all these threats, HTs are one
                    of the most significant problems. Since HTs can modify the
                    functionality of ICs, leak secret information, disable the 
                    entire chip, HTs are the security breaching point of
                    hardware security.
                    \item [*] There are three kinds of HT insertions, malicious
                    IP core insertion, design modification and layout
                    modification. The first two can be detected during the
                    functional testing phase of IC production.
                    \item [*] The layout insertion, however, can be extremely
                    small in terms of area and power consumption, which is very
                    hard to be detected during the post-silicon
                    verification.~\cite{rostami2014primer}
                    \item [*] Thus, there is a pressing need for rapid, accurate
                    and robust detection for this kind of HT insertion.
                    \item [*] This work is proposing an optical method that can
                    detect malicious tampering and insertion of HTs at the
                    fabrication stage.
                \end{itemize}
            \posterboxend
            \posterbox{Related work}
                \begin{itemize}
                    \item [*] State-of-the-art techniques for detecting HTs
                    include both destructive and non-destructive
                    approaches.~\cite{Tehranipoor2010,Karri2010}
                    \item [*] Destructive approaches requires high cost of
                    reverse engineering and precise measurements. 
                    \item [*] While the non-destructive approaches are taken the
                    form of functional testing or side-channel
                    analysis.~\cite{Chakraborty2009}
                    \item [*] Functional testing consists of system inputs and
                    verification on the outputs. However, for some HTs,
                    triggering rate is extremely small. These test can be costly
                    and time consuming.
                    \item [*] Side-channel analysis detect the HTs by analyzing
                    the physical characteristics of the chip. Such techniques
                    include power analysis, time-based analysis, emission
                    analysis, and thermal analysis.~\cite{Rad2008Iccad,
                    Alkabani2009Iccad,Potkonjak2009Dac, Sheng2012Tvlsi}
                    \item [*] One of the most popular techniques is PUF design.
                    With certain amount of challenge and response (CRP)  pairs
                    of the design, the PUF can be used as the signature of the
                    chip.~\cite{wei2012hardware}  
                    \item [*] These designs can be easily implemented but
                    requries a large amount of CRPs to have a complete coverage
                    of the measurement space.
                \end{itemize}
            \posterboxend
            \posterbox{Optical Watermarks: Design and Measurement}
                \begin{figure}
                    \centering
                    \hspace{-5mm}
                    \includegraphics[width=0.9\textwidth]{img/uni_response.pdf}
                    \vspace{-3mm}
                    \caption{{\color{black} The reflectance spectrum of functional gates and
                    fill cells, computed via FDTD simulations. The response is computed for both
                    X and Y polarizations of the illuminating field (solid and dashed lines,
                    respectively). For X polarization, the incident electric field is polarized
                    along the VDD and VSS rails. For Y polarization, the polarization is
                    perpendicular.}}
                    \vspace{-0.25in}
                    \label{fig:single_cell_response}
                \end{figure}
            \posterboxend
            \posterbox{Detection Process}
                \begin{figure}
                    \centering
                %    \hspace{-10mm}
                        \includegraphics[clip,width=0.365\textwidth]{img/layout.png}~
                        \includegraphics[clip,width=0.5\textwidth]{img/mod/Free.pdf}
                    %\vspace{-0.2in}
                    \caption{
                        (a) Physical layout of a $10\mu m \times 10\mu m$ region of the AEST100
                        hardware block. (b) Backside image (reflectance value) of the $10\mu m
                        \times 10\mu m$ region. The fill cells have the highest reflectance. 
                    }
                    \label{fig:backside-img1}
                    \vspace{-0.2in}
                \end{figure}
            \posterboxend
        \end{column}
        \begin{column}[T]{0.33\textwidth}
            \posterbox{Detection Process}
                \begin{figure}[t]
                    \begin{columns}
                    \column{0.5\textwidth}
                    \begin{center}
                        \includegraphics[clip,width=\textwidth]{img/mod/In.pdf}\\
                        (a)\\
                    \end{center}
                    \column{0.5\textwidth}
                    \begin{center}
                        \includegraphics[clip,width=\textwidth]{img/shi/In.pdf}\\
                        (b)\\
                    \end{center}
                    \end{columns}
                    \begin{columns}
                    \column{0.5\textwidth}
                    \begin{center}
                        \includegraphics[clip,width=\textwidth]{img/rep/In.pdf}\\
                        (c)
                    \end{center}
                    \column{0.5\textwidth}
                        \caption{
                         (a) Backside image (reflectance value) of the $10\mu m
                        \times 10\mu m$ region when the fill cells are replaced
                        with functional gates that constitute the Hardware
                        Trojans.  (b) Backside image (reflectance value) of the
                        $10\mu m \times 10\mu m$ region when the bottom 3 rows
                        are shifted by 5 $\mu$m to the left make room for cells
                        that constitute the HT.  }
                    \end{columns}
                        \justify
                        (c) Backside image (reflectance
                        value) of the $10\mu m \times 10\mu m$ region when the
                        functional cells are replaced by a different set of
                        cells that constitute the HT.
                \end{figure} 

                \begin{figure}[h]
                     \centering
                         \includegraphics[width=0.5\textwidth]{img/real/Free.png}~
                         \includegraphics[width=0.5\textwidth]{img/real/In.png}
                     \caption{
                         (a) Backside image (reflectance values) of a Trojan-free $50\mu m \times 50\mu
                         m$ region of the AEST100 hardware block.
                         (b) Backside image (reflectance values) of the same $50\mu m \times 50\mu
                         m$ region of the AEST100 hardware block with {\color{black}CDMA private
                         key disclosure} type of HT inserted in it.}
                     \label{imaging_result}
                 \end{figure}

                \begin{figure}[h]
                    \centering
                    \includegraphics[width=1.0\columnwidth]{img/thresholdDecision.pdf}
                        \vspace{-0.3in}
                    \label{fig:threshold}
                    \caption{Detection Error rate vs Detection Threshold.
                    \ignore{\color{red}After a Monte-Carlo simulation of
                    detection error rate against
                    detection threshold, we decide the optimized detection threshold should be
                    $0.65$ for a fixed SNR of 10.}}
                        \vspace{-0.25in}
                \end{figure}

            \posterboxend
            \posterbox{Detection Process}
                \begin{figure}[ht]
                    \begin{center}
                        \includegraphics[width=\textwidth]{img/fillercells1.pdf}\\
                        \includegraphics[width=\textwidth]{img/fillercells2.pdf}\\
                        \includegraphics[width=\textwidth]{img/fillercells3.pdf}\\
                        \includegraphics[width=\textwidth]{img/fillercells4.pdf}\\
                        \caption{Impact of process variations on the reflectance
                        signal for two different types of fill cells for various
                        wavelengths. Black lines correspond to the nominal
                        value, Red lines correspond to +10\% variation and Blue
                        lines correspond to -10\% variation. X and Y correspond
                        to X polarization and Y polarization, respectively.}
                        \label{fig:PV}
                        \vspace{-0.3in}
                    \end{center}
                \end{figure}
            \posterboxend
        \end{column}
        \begin{column}[T]{0.33\textwidth}
            \posterbox{Evaluation}

                \begin{figure}[h]
                    \centering
                    \begin{columns}
                        \column{0.5\textwidth}
                        \includegraphics[width=\textwidth]{img/snr1.pdf}\\
                        \includegraphics[width=\textwidth]{img/snr2.pdf}\\
                        \includegraphics[width=\textwidth]{img/snr3.pdf}\\
                        \column{0.5\textwidth}
                        \includegraphics[width=\textwidth]{img/snr4.pdf}\\
                        \includegraphics[width=\textwidth]{img/snr5.pdf}\\
                        \includegraphics[width=\textwidth]{img/snr6.pdf}\\
                    \end{columns}
                    \label{fig:snr}
                    \caption{
                        {\color{black}Testbench AES and PIC Trojan Detection Rate under different
                    Signal-to-Noise Ratio
                    (a) is AES100
                    (b) is AES200
                    (c) is AES1000
                    (d) is PIC100
                    (e) is PIC200
                    (f) is PIC300}
                    }
                    \vspace{-0.15in}
                \end{figure}

                \begin{table}[h]
                        %\vspace{-3in}
                        %\resizebox{\textwidth}{!}{%
                            \begin{tabular}{c c c}
                                          & Trojan Leakage        & Trojan Total Power  \\
                                Testbench & Percentage (\% of     & Percentage (\% of \\
                                          & Total leakage)        & Total power) \\
                            \hline
                                AES100    & 1.64           & 1.05 \\
                                AES200    & 0.8            & 1.0 \\
                                AES1000   & 1.64           & 1.05 \\
                                PIC100    &1.0             & 29 \\
                                PIC200    & 1.21           & 34\\
                                PIC300    &16              & 60 \\
                            \end{tabular}
                            \caption{Trojan power overhead in various hardware
                            blocks. The baseline designs of AES100, AES200 and
                            AES1000 are the same, the inserted HTs are different.
                            Similarly, the baseline design of PIC100, PIC200 and
                            PIC300 are the same but the inserted HTs are different.
                            {\color{black}The power calculation was done using
                            \textit{Nangate45nm} high performance library.}}
                            \label{tab:power}
                            %}
                \end{table}

                \begin{table}[h]
                    %\vspace{-3in}
                    %\resizebox{\textwidth}{!}{%
                    \begin{center}
                        \begin{tabular}{c c c c}
                            Testbench & Area without & Trojan     & Trojan Area      \\
                                      & Trojans      & Area       & Percentage (\%)  \\
                        \hline
                            AES100    & 274177.6     & 253.2      & 0.0923 \\
                            AES200    & 274177.6     & 169.5      & 0.0618 \\
                            AES1000   & 274177.6     & 251.1      & 0.0915 \\
                            PIC100 	  & 4215.0 & 351 & 8.33 \\
                            PIC200    & 4215.0 & 89.6 & 2.13 \\
                            PIC300    & 4215.0 & 253.2 & 6.01 \\
                        \end{tabular}
                        \caption{Area (in $\mu m^2$) occupied by various hardware blocks.}
                        \label{tab:area}
                        %}
                    \end{center}
                \end{table}
            \posterboxend
            \posterbox{Conclusion}
            \begin{itemize}
            \item [*] We propose a new technique that uses backside imaging for
            detecting HTs. 
            \item [*] We evaluate our proposed approach using a variety of AES and
            PIC blocks and show that we are able to detect extremely small HTs,
            which have power consumption that is 12 less than 2\% of the total
            chip and an area that is less than 0.1\% of the total area.
            \item [*] We also prove that our approach is robust to measurement noise
            and $\pm10\%$ process variations.
            \end{itemize}
            \posterboxend
            \posterbox{Reference}
                \bibliographystyle{nar}
                %\begin{multicols}{2}
                \bibliography{src/bib/hw-security-spectroscopy-dac2015}
                %\end{multicols}
            \posterboxend
        \end{column}
    \end{columns}

\end{document}
